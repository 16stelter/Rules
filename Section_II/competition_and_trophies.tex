\clearpage
\sffamily
{\bfseries\color[rgb]{0.4,0.4,0.4}The Competitions and Trophies}
\phantomsection
\addcontentsline{toc}{subsection}{The Competitions and Trophies}


\bigskip

{\bfseries Setup and Inspections}

\headlinebox

The competitions in the Humanoid League are preceded by a setup and inspection
period of at least 24 h.
During this time, every robot will be inspected by the league organizing
committee for compliance with the design rules detailed in Law 4.
Robots need to demonstrate the ability to walk (all sub-leagues) and stand up
from front and back (\added{KidSize} \removed{kid and teen size}).
The teams must be able to demonstrate at least one successful stand-up action
from each side during the robot inspection.
All robots will be photographed during the inspection.
A re-inspection becomes necessary after any change that could affect the
compliance to the design rules.
A re-inspection might be requested by any team leader up to 1 hour prior to a game.

\bigskip

{\bfseries Referee Duty}

\headlinebox

Each team must name at least one person who is familiar with the rules and who
might be assigned for referee duties and for the technical inspection by the
league organizing committee.

\bigskip

{\bfseries Competitions}

\headlinebox

The competitions consist of:

\begin{enumerate}
\item \removed{Soccer games} \added{Regular tournament} for KidSize (4 vs. 4),
\item \removed{Soccer games for TeenSize (3 vs. 3),}
\item \removed{Soccer games} \added{Regular tournament} for AdultSize (2 vs. 2),
\item Drop-In games for KidSize\removed{, Teen} and AdultSize
\item Technical challenges.
\end{enumerate}

\bigskip

% Combined Teams
\added{While teams do require only 1 robot to participate to the Drop-In,
they need to present a full roster of robots for the soccer games.
During robot inspection, teams have to announce how many robots they can provide.
Teams which cannot provide a full roster have to form a combined team with one
or more other teams from the same league to present a full roster.
Teams which can provide a full roster are allowed to form a combined team on their own,
but can also join other teams to form a combined team.
In case one or more teams which do not have a full roster remains but cannot
form a full roster if combining,
they are still allowed to play as a combined team.
This ensures that 
}

\bigskip

% Drop-In for Seeding + Divisions
\added{The competition starts by the Drop-In tournaments.
The scores of the combined teams during the Drop-In tournament are used to
establish a preliminary ranking.
When multiple teams join to form a combined team,
only the best score of the teams composing the combined team is used~\footnote{
  If three team A,B,C participate to the drop-in tournament,
  receive respectively $3.2$, $4.5$ and $-1.6$ points and
  participates as a combined team,
  the score of the combined team will be $4.5$.
}
If more than 16 combined teams participate in a league,
the teams are separated in two different divisions for the regular tournament:
\begin{itemize}
\item Combined teams ranked 1-4 are qualified for division A
\item Combined teams ranked 5-12 play one playoff for division A
      (Teams who loose the playoff are qualified for division B)
\item Combined teams ranked 11-20 are qualified for division B
\item Combined teams ranked 21-28 play one playoff for division B
\end{itemize}
If there are less than 16 combined teams, they all play in the same division.
}

\bigskip

\removed{Soccer games are organized in one or more round robins and playoffs. For the first round robin, the teams are assigned to groups at random.}
\added{
  For regular tournaments, each division is organized with one round robin
  composed of one or more groups followed by a knock-out tournament.
  The teams are assigned to the groups based on the preliminary ranking from the
  Drop-In competition.
}
All teams of a group play once against each other.
The round robin games may end in a draw.
In this case, both teams receive one point.
Otherwise, the winning team receives three points and the not winning team
receives zero points.

\bigskip

For the AdultSize soccer games, a specific rule for robot handlers applies. For every robot, one robot handler is allowed to stay near the robot such that the robot handler does not interfere with the game. Specifically, the robot handler:
\begin{itemize}
\item should position himself behind the robot at a distance of at least an arm length away from the robot's convex hull.
\item must not block the vision of any of the robots on the ball or goals.
\item must not block the path of any robot.
\item must not touch any robot. Touching a robot is considered an offence that is penalised by a removal penalty of the robot handler's own robot according to the laws of the game.
\item must not enter the radius of one arm length around the robot unless the robot is to be picked up or to avoid interference with the game. Violation of this rule results in a warning to the respective robot handler. After two warnings, the robot handler needs to be replaced similarly to the rule specified under ``Request for Pick-up''.
\item has to be dressed in black clothes.
\item may not communicate with the robot in any way, including verbally, while the robot is in play.
\end{itemize}
\color{black}

\bigskip

After games of a round robin have been played, the teams of a group are ranked based on (in decreasing priority): 

\begin{enumerate}
\item the number of earned points,
\item the goal-difference,
\item the absolute number of goals,
\item the result of a direct match,
\item the time needed to score a penalty kick into an empty goal (up to five alternating attempts to score, until at least one team scored),
\item the drawing of a lot.
\end{enumerate}

\bigskip


At least two teams of every group will enter the next round robin or the playoffs.

\bigskip

In the knock-out games of a tournament two further equal periods of 5 minutes each are played if the game is not decided after the regular playing time.  After consulting the team captains, the referee can decide to skip the extra time and continue the game immediately with the five alternating penalty kick trials. If both teams agree, the regular penalty kick trials may be skipped and the extended penalty kick procedure applies instead.

\bigskip


\removed{The game plan needs to be announced prior to the random assignment of teams to groups.}


\bigskip

\removed{Participation in the drop-in challenge is mandatory to advance to the quarter final. If a team advances to the quarter final without participating in the drop-in challenge it is replaced by the team that would have had the next best prospects to reach the quarter finals in direct comparison.}

\bigskip

{\bfseries Forfeits}

\headlinebox

A team that forfeits is disqualified from the competition. Forfeiting is defined as refusing to make a good faith effort to participate in a scheduled game\footnote{If robots are broken, then they should be placed on the field anyway as an indication that the team is willing to participate.}.
\begin{itemize}
\item If a team chooses to forfeit a match in the round robin games the other team plays on an empty goal.
\item If a team chooses to forfeit in a knock-out game before the quarter final, the other team continues in the competition.
\item If a team chooses to forfeit in the quarter finals, it will be replaced by the runner up team in the round robin group that included the forfeiting team.
\item If a team chooses to forfeit in the semi finals or the game for 3rd and 4th place, it will be replaced by the team that lost to the forfeiting team in the quarter finals.
\item A team forfeiting the final match should announce its decision at least 30 minutes before the start of the 3rd vs 4th final.  The league organization committee may impose a one year disqualification of the team and its members in case of avoidable delayed announcements.
\item If a team chooses to forfeit the final after the game for 3rd and 4th place began, it will be replaced by the 3rd place winner, and the 4th place winner will be 3rd place. No new 4th place will be selected.
\item If a team chooses to forfeit the final before the game for 3rd and 4th place, it will be replaced by the team that lost to the forfeiting team in the previous semi-finals. The team that lost to the forfeiting team in the previous semi-finals (i.e. the one initially being qualified for the game for 3rd and 4th place) will be replaced for the game for 3rd and 4th place by the team that lost to it in the respective quarter final.
\end{itemize}

\clearpage
{\bfseries Gameplay}

\headlinebox
 
{\bfseries Kick-off procedure}

\begin{itemize}
\item The referee gives the signal ``READY'' that all robots have to reach their own half of the field. 
\item After a period between 15 and 45 seconds the referee gives the signal ``SET''.
      The referee calls robots illegally positioned to be removed with exception
      of the goal keeper and a striker for the team having kick-off that may be
      placed manually by the respective robot handler.
      The robot playing the goal keeper has to be announced before the beginning
      of the game.
      The striker can be chosen after the end of the ``READY'' phase among all
      robots of the team (even those being called for illegal positioning).
      A manually placed goal keeper has to be within the team's own goal area
      touching the goal line.
      A manually positioned striker can take any legal position within the team's own half.
      Removed players and all other players outside the field of play are placed
      on the touch line close to the penalty mark of the respective team's side,
      facing the opposite touch line, to enter the field from there upon the ``PLAY'' signal.
      Illegally positioned players do not suffer a removal penalty and are
      allowed to enter the field as soon as the game starts.
\item The opponents of the team taking the kick-off are outside the centre circle until the ball is in play.
\item The ball is stationary on the centre mark. 
\item The referee gives the signal ``PLAY'' or whistles.
\item The ball is in play when it is kicked and clearly moves as determined by
      the referee or 10 seconds elapsed after the signal.
\end{itemize}

Robots being able to autonomously reposition themselves can take any position on the field that is consistent with the above requirements. It is not allowed to manually (re-)position an autonomously positioned robot that took a legal position, except for the goal keeper and the striker for kick-off or dropped ball, if applicable.
 
For initial kick-off (to initially start or restart after a half-time interval), robots can be placed anywhere on the touch lines or goal lines on the respective team's own side outside of the goal, facing the opposite touch line or goal line, to autonomously enter the field from there. For other kick-off situations robots need to position themselves from the the position they were when the game was stopped.

Similar to a kick-off situation, one striker robot for each team may be placed manually for a dropped ball, if both teams can not position themselves automatically. The striker for dropped ball may be positioned at any legal position within the team's own half outside of the centre circle.

\bigskip

If one or both of the teams in a match have permission to use a manual startup procedure, the referee gives the two signals ``SET'' and ``PLAY'' with an interval of exactly
10 seconds. Robot players without remote start capability may be started on the field after the ``SET'' signal. They may not move before the ``PLAY'' signal was given
by the referee.
Robots with autonomous positioning ability are given between 15 and 45 seconds
for re-positioning after a goal has been scored by one of the teams before the
``SET'' signal for kick-off is given by the referee.
All human team members must leave the field of play immediately after the
``SET'' signal and before the ``PLAY'' signal.

A team which uses manual kick-off and not kick-off with the referee box is punished by a penalty of having to wait for 15 seconds after the signal ``PLAY'' before they are
allowed to locomote. If such a team has kick-off then at the signal ``PLAY'' the referee immediately moves the ball from the kick-off position to a position somewhere on
the centre line and outside of the centre circle and the ball is in play. This rule does not apply if the referee box is out of commission.

\bigskip


{\bfseries Free-kick procedure}

\begin{enumerate}
\item The referee blows the whistle, announces the offence and the free kick
      (e.g 'Pushing Red - Direct Free-Kick Blue')
\item The assistant referee who operates the game controller clicks on "Direct / Indirect Free Kick" Blue/Red.
\item The referee places the ball depending on the call and announces "Free Kick Ready".
      Until the referee announced the free kick to be ready robots may move
      their head to track the ball but must otherwise remain stationary.
      Robots which are significantly moving during this phase are removed from the field.
\item The assistant referee who operates the game controller clicks on
      Prepare Direct / Indirect Free Kick" Blue/Red.
      The player taking the free kick has up to 30 seconds to position
      themselves for the free kick.
      The team taking the free kick may announce that the player is ready to
      take the free kick at any point.
      Opponent robots must move to a position at least 0.75 m for KidSize and
      \removed{TeenSize and} 1.5m for AdultSize away from the ball.
      They are guaranteed at least 15 seconds to move away from the ball.
      They may take up to 30 seconds if the team taking the free kick has not
      announced their robot is ready to take the kick off.
\item The assistant referee announces that 15 seconds and, if applicable, 30 seconds are over.
\item The referee may decide to execute the free kick any time between 15 and 30 seconds
      have passed depending on the call of the team taking the free kick and the position of
      the opponent robots.
      The referee may also allow the free kick to be executed before 15 seconds have passed
      if the team taking the free kick have announced their robot is ready and if no opponent is
      illegally positioned.
\item Any opponent robot still illegally positioned is considered as an
      incapable player and must be removed from the field for 30 seconds
      removal penalty.
\item When the referee decides to execute the free kick and all opponent robots are
      legally positioned or have been removed from the field,
      the referee blows the whistle.
      Then the assistant referee who operates the game controller clicks on
      "Execute Direct / Indirect Free Kick" Blue/Red.
      The team that was awarded the free-kick can kick now.
\item The ball is in play after it has been
      kicked and clearly moves as determined by the referee or after 10 secs.
\end{enumerate}

The distance between the ball and the player of the opponent team is measured between the point on the convex hull of the robot and the point on the outside of the ball which are most close together.

\color{black}


The referee blows the whistle, announces 'Free-Kick' blue or red and then places
the ball depending on the call.
The assistant referee who operates the game controller clicks on
``Direct / Indirect Free Kick'' Blue/Red.
The referee places the ball depending on the call and announces
``Free Kick Ready'' and the assistant referee clicks on
`` Prepare Direct / Indirect Free Kick'' Blue/Red.
The player taking the free kick has up to 30 seconds to position themselves for
the free kick.
The team taking the free kick may announce that the player is ready to take the
free kick at any point.

Players are guaranteed at least 15 seconds to move away from the ball.
They may take up to 30 seconds if the team taking the free kick has not announced
their robot is ready to take the kick off.
Any opponent robot still illegally positioned is considered as an incapable
player and must be removed from the field for 30 seconds removal penalty.
The referee may decide to execute the free kick before 15 seconds have passed if
the team taking the free kick have announced their robot is ready and if no
opponent is illegally positioned.

Once the free kick can be executed, the referee blows the whistle and the
assistant referee clicks on ``Execute Direct / Indirect Free Kick'' Blue/Red.


\bigskip


{\bfseries Request for Pick-up}

A robot handler may request to pick-up a robot if and only if a robot is in a
dangerous situation that is likely to lead to physical injuries.
If a robot handler touches a robot without the allowance of the referee,
the respective robot receives a yellow card and the robot handler an official warning.
In AdultSize, a request for pick-up is implicitly granted if the robot is
visible unstable or about to fall.
After two warnings for robot touching, the robot handler may not serve as a
robot handler again for the rest of the game and needs to be replaced by a
different person from the team immediately.
In case of repeated violations throughout the tournament, the Technical
Committee can decide to permanently disallow a certain person from serving as
a robot handler for the rest of the tournament.

\bigskip

{\bfseries Incapable Players}

Players not capable of play (e.g. players not able to walk on two legs,
players not able to stand, or players with obvious malfunctions)
are not permitted to participate in the game.
They must be removed from the field.
It is up to the referee to judge whether a player is capable of play.
The referee may ask the team leader of a player suspected to be incapable of
play to demonstrate playing ability at any time.
A field player that is not able to get back into a stable standing or walking
posture from a fall within 20 seconds will be removed from the field for 30
seconds removal penalty.
It has to enter the field from the team's own half of the field close to the
penalty mark facing the opposite touch line, as indicated by the referee.
If the ball is within a radius of 0.5 m around the goal keeper inside the goal area,
the goal keeper has to show active attempts to move the ball out of this radius.
If no attempt is shown for 20 seconds, the goal keeper is considered to be an
inactive player and receives a 30 second removal penalty.

\bigskip

{\bfseries Substitutions}

Up to two players per game can be substituted by other players of the same team.
A substituted robot can come back in play but it counts as an additional substitution.
The referee must be informed prior to the substitution.
A substitute only enters the field after the player being replaced left the
field and after receiving a signal from the referee.
Any of the other players may change places with the goalkeeper,
provided that the referee is informed before the change is made and that the
change is made during a stoppage of the match.
Changing places/roles between a field player and a goalie does not count as
substitution.

\bigskip

{\bfseries Temporal Absence}

Servicing robots on the playing field is not permitted.
A robot may be taken out of the field for service,
after receiving permission from the referee.
Taking out a robot for service does not count as a substitution.
A serviced robot may not come into play again before 30 seconds elapsed after it
was taken out.
It has to enter the field from the team's own half of the field close to the
penalty mark facing the opposite touch line, as indicated by the referee.
The same rules as for \textit{Removal Penalty} apply.

\bigskip

{\bfseries Manual Untangle of Robots}

If entangled robots fail to untangle themselves, the referee might ask designated robot handlers of both teams to untangle the robots. Untangling must not make
significant changes to robot positions or heading directions. Untangled robots must be laid on the ground not closer than 50cm to the ball and in a way not gaining an advantage.

\bigskip


{\bfseries Removal Penalty}

\begin{itemize}
\item Time penalties of 30 seconds for players are called by the referee. When a penalty is called, the designated robot handler has to remove the robot as soon as possible and by that interacting as little as possible with the game. 
\item The referee and assistant referees are in charge of timing the penalties and notifying the teams to put back their robots to play.
\item A field player or goal keeper suffering a time penalty will be removed from the field and is only allowed to re-enter the field from the team's own half of the field close to the penalty mark facing the opposite touch line, as indicated by the referee.
\item After the robot has been placed at the position indicated by the referee
      and with both feet entirely outside the field of play the robot handler
      announces to the assistant referee that the robot is ready to get back in.
      The 30 seconds penalty start counting from the point of announcement.
      From this point onwards the robot handler may not touch or interfere with
      the robot in any other way (including button presses).
      If any part of the robot touches the field of play (including touch lines)
      or the robot handler touching the robot before the 30 seconds expired,
      the time is reset.
\item The assistant referee operating the GameController will:
\begin{itemize}
\item Penalize the robot as soon as the referee calls the penalty.
\item Marks the penalty time counting down as soon as the robot handler
      announced the robot being ready to walk in
\item Resets the penalty time whenever the robot handler touches the robot or
      the robot touches the field of play
\end{itemize}
\item The penalty is automatically removed after 30 seconds of penalty have expired.
\end{itemize}

\bigskip

{\bfseries Timeouts}

A team may call for a timeout before kick-off after a goal was scored,
the start of a new half, or a drop ball was called and before a penalty shoot-out.
During a timeout robots may be serviced.
Each team may take at most one timeout per period during regular game time and
one additional timeout during all of the extended time and penalty shoot-out.
If a team is not ready to resume the game when the referee wants to start the game,
it has to take a timeout.
If there is no timeout left, the referee will start the game anyway.
A timeout ends automatically after 120 s.
A timeout also ends when the team signals its end to the referee.

\bigskip

{\bfseries Referee Timeouts}

The head referee may call a timeout before kick-off after a goal was scored,
the start of a new half, or a drop ball was called and before a penalty
shoot-out if they deem it necessary.
A referee timeout should only be called in dire circumstances -
one example might be when the power to the wireless router is down.
However, when and whether to call a referee timeout is left up to the head referee.

Referees may call multiple timeouts during a game if needed. Teams may do anything during these timeouts, but they must be ready to play 2 minutes after the referee begins a timeout. The referee should end the timeout once they believe the circumstance for which the timeout was called has been resolved. In cases where the circumstance for which the timeout was called is not resolved within 10 minutes, the Technical Committee should be consulted regarding when/if play should continue.

The team who would have kicked off if the timeout had not been called shall kickoff when the game resumes.

\bigskip

{\bfseries Disciplinary sanctions}

Yellow and red cards given to robots only accumulate for the current game and are cleared again after the end of each game. Warnings against robot handlers and/or teams have to be reported to the Technical Committee after each game. They are recorded and accumulated for the whole tournament. 




\bigskip

{\bfseries Drop-In Games}

\headlinebox

{\bfseries Organisation}

Each participating team will contribute one drop-in player for each drop-in game.
The drop-in player may be chosen from all available robots of the team and does
not have to be the same in all drop-in games.
Each drop-in player will compete in games with many different teams composed of
randomly chosen drop-in players.
In each game, the opponent will be a similarly composed team of randomly
selected drop-in players.
The exact number of games played by each drop-in player depends on the number of
teams that participate in the competition.
A minimum of \removed{3} \added{4} drop-in games will be played by every team.

\bigskip

The drop-in players will be allocated to teams randomly at least 24 hours before
the first game takes place.
The allocation to teams for the individual games is randomly chosen and changes
for every single game.
\added{Some teams might play one more Drop-In game than others.}
\removed{Drop-in players that reached the minimum number of games will not be
considered for drawing unless needed to fill up a team,
to have games for remaining players with less than 3 games.}

\bigskip

{\bfseries Rules}

All normal game rules apply to this competition. The only exceptions are:
\begin{enumerate}
\item The games are played with 5 players in a KidSize team
      \removed{, 4 players in a TeenSize team} and 3 players in an AdultSize team.
      If there is an insufficient number of participants,
      games may be played 4 vs. 4 or 3 vs. 3 for KidSize
      \removed{, 3 vs. 3 or 2 vs. 2 for TeenSize,} or 2 vs. 2 for AdultSize.
\item Games may end in a draw.
\item Each of the players has a jersey number from the set {1, 2, 3, 4, 5},
      \removed{resp. {1, 2, 3, 4}} resp. {1, 2, 3}.
\item Drop-in teams will wear the blue and red team colours.
\item In AdultSize, one robot handler per competing robot is allowed.
      In KidSize \removed{and TeenSize}, the teams have to agree on one robot handler per team.
\end{enumerate}

Removal of incapable players has to be enforced strictly.

\bigskip
 
{\bfseries Communication}

Teams are strongly encouraged to implement the mitecom team communication protocol which is available at \\
\textcolor[rgb]{0.0,0.0,0.49803922}{https://github.com/RoboCup-Humanoid-TC/mitecom}

\bigskip
 
{\bfseries Selection of the Referees}

Referees will be drawn from the remaining participating teams, or if needed,
due to a low number of teams, the TC and OC will provide referees.
Referees for a match may be picked among the pool of available referees from any size class.

\bigskip
 
{\bfseries Scoring}

When a goal is scored, all players of the scoring team on the field receive 1 point and the player who scored the goal receives an additional 1 point if it was not an own goal. A robot is considered to be on the field if both feet of the robot are fully inside the field area. The player who scored the goal receives the points regardless of its position on the field. Incapable players, penalized players, players outside the field or players having been removed for any other reason, e.g. service, will receive no point. When a goal is suffered, all players of the team suffering the goal receive -1 points, including incapable players, penalized players or players having been removed for any other reason, e.g. in service. Points from all games are summed up. For players who played one or more games more than the others, only the points of those games with the higher scores are considered.

Drop-in players are initially ranked according to the accumulated points. If there are ties, the tied players are ranked according to the arithmetic mean, the number of games played, the maximum points awarded in a single game and the number of goals scored (in this order). The three most highly ranked players of a sub-league receive a 'Best Player' certificate.

If the three best drop-in players can not be identified with the scheme provided above, an additional game with players drawn group wise (from the groups of equally high ranked players with a total number of points greater than zero) from the highest-ranked players is to be played.

If there are still ties a penalty shoot out among the equally high ranked drop-in players with a total number of points greater than zero takes place.

\bigskip 
 
{\bfseries Example}
 
There are 20 participating players for 4 KidSize Drop-In teams (A, B, C, D). Games shall be A-B, A-C, A-D, B-C, B-D, C-D. For the first game (A-B), Player 1 is randomly drawn for Team B and plays the game with the other team members. For the second game (A-C), player 1 is not drawn. For the third game (A-D), player 1 is drawn for team A and plays the game. For the fourth game (B-C), player 1 is drawn to team C and plays the game. Now player 1 will be removed from the drawing, because the number of 3 games was reached. If there would be only 19 participants, player 1 may be drawn for the final game.

Assuming teams with player 1 to score a single goal in every game (with player 1 on the field), then player 1 will have a score of 3 and an arithmetic average of 1.

\bigskip

{\bfseries Technical Challenges}

\headlinebox

The technical challenges consist of:

\begin{enumerate}
\item Push Recovery (KidSize\removed{, TeenSize} and AdultSize)
\item Goal Kick from Moving Ball (KidSize\removed{, TeenSize} and AdultSize)
\item High Jump (KidSize \removed{and TeenSize} and AdultSize)
\item High-Kick (KidSize\removed{, TeenSize} and AdultSize)
\end{enumerate}

For details on the technical challenges, please refer to Section III of this document.

\bigskip

\newpage
{\bfseries Best Humanoid Award}

\headlinebox
 
The teams of the Kid\removed{, Teen} and Adult size classes that have
participated in the drop-in competition are ranked in separate lists to
determine the overall best humanoid.

The ranking is based on the aggregated number of points earned in the individual competitions.

\bigskip

The points earned in the technical challenge are used directly.

\bigskip

For the Drop-In challenge the winner receives 30 points. The second best team receives 21 points. The third best team receives 15 points.


\bigskip

For the soccer games the winner receives 60 points. The second best team receives 42 points. The third best team receives 30 points.

\bigskip

The teams ranked first in the KidSize\removed{, TeenSize} and AdultSize lists
are candidates for the best humanoid.
The final ranking between the three candidates is determined by the points
earned in the individual competitions as stated above.
The best KidSize\removed{, TeenSize} or AdultSize team with the most overall
points wins the best humanoid award.
If there are ties, the average goals scored per game is used to determine the
ranking of the three eligible candidates.
If there are still ties, the president of the RoboCup Federation breaks the tie.

\bigskip

{\bfseries Trophies}

\headlinebox

A trophy is awarded to the winner of the soccer tournament in each of the individual size classes and technical challenges. In case of less than 3 teams participating in a size class no trophies will be given in this class.

\bigskip

A trophy is awarded to the teams second and third in the KidSize 4-4 soccer game
\removed{, the TeenSize 3-3 soccer games} and the AdultSize 2-2 soccer game.
In case of less than 5 teams participating in a size class the team ranked third
will be awarded a certificate instead of a trophy.
In case of less than 4 teams participating in a size class the team ranked
second will also be awarded a certificate instead of a trophy.
The final number of trophies awarded will be decided by the RoboCup Federation
based on the number of actually participating teams.

\bigskip

Certificates are awarded to the teams second and third in the technical challenges,
to the team\removed{s} ranked second \removed{and third} in the Best Humanoid list
and a 'Best Player' certificate to the three most highly ranked Drop-in players.
A team can only receive a 'Best Player' certificate if their total amount of
points is greater than 0.

\bigskip

The Best Humanoid Award is awarded to the team ranked first in the Best Humanoid list,
it can either be a trophy or a certificate.

\bigskip

{\bfseries Conflict Resolution}

\headlinebox

It is the responsibility of the team leader to inspect the other team's robot an hour in advance of a game. Any concern regarding the rule compliance of any of the robots, including the amount, size and colour of the team markers, must be brought to the attention of the referee an hour in advance of the game. If the referee is unavailable, they have to be brought to the attention of the Technical Committee instead.

\bigskip

Doubts concerning a serious violation of any rule during a specific game must be brought up to a member of the Technical Committee and investigated before signing the result sheet. By signing the result sheet, a team agrees that the result came off in a fair game. If a team brings up an official concern to the Technical Committee, a meeting of the Technical Committee must be called as soon as possible. If the team of a member of the Technical Committee is directly involved in the game in question, the respective member is excluded from the meeting. At least three members of the Technical Committee need to be part of the meeting and the decision process. If less than three members of the Technical Committee are available, members of the Organizing committee or, if necessary, Trustees or members of committees from other leagues have to be called into the meeting. Members of these meetings may request to inspect the hardware and software of any team involved in the issue. If serious violations of rules are detected, the committee may, among others, decide to invalidate the result of the game in question or take disciplinary actions against a team as defined in Law 5, depending on the severity of the rule violation. The decision of the committee meeting need to be announced to the whole league.

\bigskip

{\bfseries Acknowledgements}

\headlinebox

These rules evolved from previous versions of the RoboCup Humanoid League rules. We would like to thank Henry Yen for the conversion of the rules into a basic LaTeX version in 2010. The 2008 version of the rules was compiled by Pasan Kulvanit and Oskar von Stryk, the 2007 version of the rules by Emanuele Menegatti and the 2006 version by Sven Behnke, who did a remarkable job improving the rule document and gearing it towards the FIFA Laws of the Game. The improvements of the 2005 version were compiled by Norbert Michael Mayer. Philipp Allgeuer significantly contributed to the conversion of the 2017 rule book to LaTeX.  Other input came earlier from the rules of the RoboCup MiddleSize and Four-Legged Leagues.

The rules were continuously discussed within the technical committee of the humanoid league and also on the humanoid league mailing list. The following members of the technical committee for 2019 were responsible for the rule evolution: Jacky Baltes, Hafez Farazi, Reinhard Gerndt, Ludovic Hofer, Maike Paetzel, Soroush Sadeghnejad and Michael Sattler. Special thanks go to Martin Friedmann, Sebastian Mielke and Timon Giese for the contribution of several figures and to Philipp Allgeuer for his contributions to the 2019 version of the rules.
