\clearpage
\sffamily
{\bfseries\color[rgb]{0.4,0.4,0.4}
General Rules for Technical Challenges}
\phantomsection
\addcontentsline{toc}{subsection}{General Rules for Technical Challenges}

\bigskip

The technical challenges consist of the following individual challenges:

\begin{itemize}
\item Part A: Push Recovery
\item Part B: Goal Kick from Moving Ball
\item Part C: High Jump 
\item Part D: High Kick
\end{itemize}

\bigskip

Only the robots used for the soccer games are allowed to participate in the technical challenges. \added{At any point, two robots are considered active during the technical challenges. One player can be substituted by another player of the same team.} No hardware modifications of the robots are allowed for the Technical Challenge (i.e., a robot cannot be modified from the configuration it had in the soccer games). 

\bigskip

The team scheduled for the Technical Challenge must have access to the field five minutes prior to the scheduled starting time. The referee will give the start signal at the scheduled time.

\bigskip

{\bfseries Method of scoring}

\headlinebox
 
The Technical Challenge consists of four parts A, B, C and D. Each of the parts can be attempted multiple times, in any order. The team might terminate a trial at any time, in order to reattempt the same part or switch to another part of the challenge. A trial terminates automatically when 25 minutes elapsed after the referee gave the start
signal. This concludes the Technical Challenge for the team. The time is taken for each of the trials, if completed successfully. Ranking in the individual challenges is determined according to the rules defined
for each individual challenge. For each part, the highest ranked team receives 10 points. The second team receives 7 points. The third team receives 5 points. All other teams who successfully managed this part of the challenge receive 3 points.

\bigskip

{\bfseries Robot Handlers During the Technical Challenge}

\headlinebox

During an ongoing trial of a technical challenge the robot handler is not allowed to interact with the robot's sensors in any way. In the moment the handler either touches the robot or interacts with the robot in any way, the trial is finished and counted as unsuccessful.

\begin{itemize}
\item In KidSize and TeenSize robot handlers are not allowed to enter the field vicinity of the robot during a trial, unless the referee asks them to remove a robot.
\item In AdultSize one robot handler is permitted to stay on the field near the robot during a trial.
\end{itemize}