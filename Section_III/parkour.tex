\clearpage
\sffamily
{\bfseries\color[rgb]{0.4,0.4,0.4} Part C: Parkour}
\phantomsection
\addcontentsline{toc}{subsection}{Part C: Parkour}


\bigskip
The goal of the Parkour TC is to have a robot going up one platform, staying on top of it, and going down without falling on the ground.

The minimum height must be at least $1/5^{th}$ of the robot height and must be a multiple of 5 cm.

The robot can touch the platform with its limbs (arms and legs), but no other part of the robot is allowed to touch it.

The platform will have an area of approximately 60 {\texttimes} 60 cm.

\bigskip

{\bfseries Run Setup}

\smallskip

The initial setup of a run is as follows:

\begin{enumerate}

\item The robot is placed in front of the platform as high as chosen by the team taking the technical challenge.

\item The referee blows the whistle to start the run.

\item Teams may start the robot manually by pressing a button when the run starts. But the robot must not be touched after the referee blew the whistle. 

\item A chronometer is started when the referee blows the whistle.
\end{enumerate}

{\bfseries Run evaluation}

\smallskip

The chronometer is stopped when the run ends. The causes for the end of a run and
the possible results are as following:
\begin{itemize}
\item \textit{Failure}
  \begin{itemize}
    \item The robot is not able to go up the platform.
    \item The robot falls without going up the platform.
        \item The robot touches the platform with a part of the body that is not a limb.
     \end{itemize}
\item \textit{Partial success}
  \begin{itemize}
    \item The robot is able to go up the platform, with both the feet touching thread (the top of the platform), but falls before being able to go down.
     \item The robot is able to go up the platform, with both the feet touching thread (the top of the platform), and, to prevent the robot from falling, a human handler touches the robot at this moment.
  \end{itemize}
\item \textit{Success}
  \begin{itemize}
    \item The robot goes up and down the platform, without falling, and stays up, without moving, for 5 seconds.
  \end{itemize}
\end{itemize}

{\bfseries Trials and ranking}

\smallskip

Teams are ranked based on their best 2 consecutive results.

A trial is considered as successful if at least 2 runs in a row resulted in \textit{Success}. A trial is considered as
partially successful if at least 2 runs in a row resulted in \textit{Success} or \textit{Partial success}.

The teams are ranked according to the following criteria on their best batch:
\begin{enumerate}
\item The maximum height the robot successfully managed to achieve on a \textit{Successful} trial divided by the height of the robot.
\item The maximum height the robot managed to achieve on a \textit{Partially Successful} trial divided by the height of the robot.
\item Average time of a \textit{Successful} trial.
\item Average time of a \textit{Partially Successful} trial.
\end{enumerate}
