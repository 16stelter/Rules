\clearpage
\sffamily
{\bfseries
\textcolor[rgb]{0.4,0.4,0.4}{Law 15 -- The Throw-In} }
\phantomsection
\addcontentsline{toc}{section}{Law 15 -- The Throw-In}


\bigskip

A throw-in is a method of restarting play.

\bigskip

A throw-in is awarded to the opponents of the player who last touched
the ball when the whole of the ball crosses the touch line, either on
the ground or in the air.

\bigskip

A goal cannot be scored directly from a throw-in.

\bigskip

{\bfseries Procedure }

\headlinebox 

If the ball leaves the field it will be replaced on the field by the
referee or an assistant referee. There is no stoppage in play.

The positions for replacement of the ball are described in the
following with respect to the touch lines and always meant to be in a
distance of about 40cm orthogonal to the position on the touch line and
inwards to the playing field. 

If the whole of the ball passes over a touch line then the assistant
referee will replace the ball back on the field on the same side of the
field as the ball went out of play. The ball will be replaced in one of
three positions: 

\begin{itemize}
\item If the referee cannot determine which robot was the last to touch
the ball before it left the field, then the ball is replaced directly
in from the point at which the ball left the field. 
\item Otherwise, the ball is placed one meter back from the point it
went out, where ``back'' is defined as being towards the goal of the team that last touched the ball. 
\end{itemize}

In any case, the ball cannot be placed closer than the length of the
goal area to either end of the field. 

Balls are deemed to be out based on the team that last touched the ball,
irrespective of who actually kicked the ball.

After placing the ball, the ball is in play immediately.

\bigskip

\greyed{
(replaces: At the moment of delivering the ball, the thrower:

\begin{itemize}
\item faces the field of play
\item has part of each foot either on the touch line or on the ground outside
the touch line)
\item holds the ball with both hands)
\item delivers the ball from behind and over his head)
\item delivers the ball from the point where it left the field of play
\end{itemize}

\bigskip

All opponents must stand no less than 2 m (2 yds) from the point at
which the throw-in is taken.

\bigskip

The ball is in play when it enters the field of play.

\bigskip

After delivering the ball, the thrower must
not touch the ball again until it has touched another player.)}

\simplify{
\bigskip

{\bfseries Infringements and sanctions}

\headlinebox

\greyed{
(suspended: Throw-in taken by a player other than the goalkeeper

If, after the ball is in play, the thrower touches the ball again
(except with his hands) before it has touched another player:

\begin{itemize}
\item an indirect free kick is awarded to the opposing team, to be taken from
the place where the infringement occurred (see Law 13 -- Position of
free kick)
\end{itemize}

\bigskip

If, after the ball is in play, the thrower deliberately handles the ball
before it has touched another player:

\begin{itemize}
\item a direct free kick is awarded to the opposing team, to be taken from the
place where the infringement occurred (see Law 13 -- Position of free
kick)
\item a penalty kick is awarded if the infringement occurred inside the
thrower's penalty area
\end{itemize}

\bigskip

Throw-in taken by the goalkeeper

If, after the ball is in play, the goalkeeper touches the ball again
(except with his hands), before it has touched another player:

\begin{itemize}
\item an indirect free kick is awarded to the opposing team, to be taken from
the place where the infringement occurred (see Law 13 -- Position of
free kick)
\end{itemize}

\bigskip

If, after the ball is in play, the goalkeeper deliberately handles the
ball before it has touched another player:

\begin{itemize}
\item a direct free kick is awarded to the opposing team if the infringement
occurred outside the goalkeeper{\textquoteright}s penalty area, to be
taken from the place where the infringement occurred (see Law 13 --
Position of free kick)
\item an indirect free kick is awarded to the opposing team if the
infringement occurred inside the goalkeeper{\textquoteright}s penalty
area, to be taken from the place where the infringement occurred (see
Law 13 -- Position of free kick)
\end{itemize}

\bigskip

If an opponent unfairly distracts or impedes the thrower:

\begin{itemize}
\item he is cautioned for unsporting behaviour
\end{itemize}

\bigskip

For any other infringement of this Law:

\begin{itemize}
\item the throw-in is taken by a player of the opposing team)
\end{itemize}
}
}