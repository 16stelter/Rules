\clearpage
\sffamily
{\bfseries
\textcolor[rgb]{0.4,0.4,0.4}{Law 15 -- The Throw-In} }
\phantomsection
\addcontentsline{toc}{subsection}{Law 15 -- The Throw-In}


\bigskip

A throw-in is a method of restarting play.

\bigskip

A throw-in is awarded to the opponents of the player who last touched the ball when the whole of the ball crosses the touch line, either on the ground or in the air.


\bigskip

A goal cannot be scored directly from a throw-in:
\begin{itemize}
\item if the ball enters the opponents' goal - a goal kick is awarded
\item if the ball enters the thrower's goal - a corner kick is awarded
\end{itemize}

\bigskip

{\bfseries Procedure }

\headlinebox 

If the ball leaves the field it will be replaced on the field by the referee or
an assistant referee.
If the whole of the ball passes over a touch line the ball is placed on the
touch line directly at the point at which the ball left the field.

\bigskip

Balls are deemed to be out based on the team that last touched the ball,
irrespective of who actually kicked the ball.

\bigskip

After placing the ball, the same procedure and rules of executing a indirect free kick apply.
Robots are also allowed to perform the throw-in with their hands, in this case:
\greyed{(replaces: At the moment of delivering the ball, the thrower:)}

\begin{itemize}
\item faces the field of play (in the physical competition only)
\item has part of each foot either on the touch line or on the ground outside
      the touch line
\item holds the ball with both hands in the physical competition and at least one hand in the virtual competition
\item delivers the ball from behind and over his head (in the physical competition only)

\greyed{\item (suspended: delivers the ball from the point where it left the field of play)}
\item releases the ball within 10 seconds (in the virtual competition only)
\end{itemize}

If a robot tries to perform a throw-in with hands and fails to respect the
rules, a free-kick is awarded to the opponent team.

\bigskip

\greyed{
(suspended: All opponents must stand no less than 2 m (2 yds) from the point at
which the throw-in is taken.

\bigskip

The ball is in play when it enters the field of play.

\bigskip

After delivering the ball, the thrower must
not touch the ball again until it has touched another player.)}

\simplify{
\bigskip

{\bfseries Infringements and sanctions}

\headlinebox

\greyed{
(suspended: Throw-in taken by a player other than the goalkeeper

If, after the ball is in play, the thrower touches the ball again
(except with his hands) before it has touched another player:

\begin{itemize}
\item an indirect free kick is awarded to the opposing team, to be taken from
the place where the infringement occurred (see Law 13 -- Position of
free kick)
\end{itemize}

\bigskip

If, after the ball is in play, the thrower deliberately handles the ball
before it has touched another player:

\begin{itemize}
\item a direct free kick is awarded to the opposing team, to be taken from the
place where the infringement occurred (see Law 13 -- Position of free
kick)
\item a penalty kick is awarded if the infringement occurred inside the
thrower's penalty area
\end{itemize}

\bigskip

Throw-in taken by the goalkeeper

If, after the ball is in play, the goalkeeper touches the ball again
(except with his hands), before it has touched another player:

\begin{itemize}
\item an indirect free kick is awarded to the opposing team, to be taken from
the place where the infringement occurred (see Law 13 -- Position of
free kick)
\end{itemize}

\bigskip

If, after the ball is in play, the goalkeeper deliberately handles the
ball before it has touched another player:

\begin{itemize}
\item a direct free kick is awarded to the opposing team if the infringement
occurred outside the goalkeeper's penalty area, to be
taken from the place where the infringement occurred (see Law 13 --
Position of free kick)
\item an indirect free kick is awarded to the opposing team if the
infringement occurred inside the goalkeeper's penalty
area, to be taken from the place where the infringement occurred (see
Law 13 -- Position of free kick)
\end{itemize}

\bigskip

If an opponent unfairly distracts or impedes the thrower:

\begin{itemize}
\item he is cautioned for unsporting behaviour
\end{itemize}

\bigskip

For any other infringement of this Law:

\begin{itemize}
\item the throw-in is taken by a player of the opposing team)
\end{itemize}
}
}
