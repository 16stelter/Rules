\clearpage
\sffamily
{\bfseries\color[rgb]{0.4,0.4,0.4}{Law 10 -- The Method of Scoring} }
\phantomsection
\addcontentsline{toc}{subsection}{Law 10 -- The Method of Scoring}

\bigskip

{\bfseries Goal scored }

\headlinebox

A goal is scored when the whole of the ball passes over the goal line,
between the goalposts and under the crossbar,
provided that no infringement of the Laws of the Game has been committed
previously by the team scoring the goal.

\bigskip

If the kicking robot is touched by the robot handler in the physical competition or removed from the game due to a removal penalty before the ball passes the goal line,
the goal does not count.
The restart of the play will be a goal kick for the opponents team.
If another robot of a team is touched by the robot handler in the physical competition or removed from the game before the ball
passes the goal line and it is not the kicker, the goal counts.

\bigskip

Note that if a penalized robot scores a goal against its own team, the
  goal is still considered as valid.

\bigskip

{\bfseries Winning team}

\headlinebox

The team scoring the greater number of goals during a match is the winner. If both teams score an equal number of goals, or if no goals are scored, the match is drawn. 

\bigskip

{\bfseries Competition rules }

\headlinebox

When competition rules require there to be a winning team after a match or home-and-away tie, the only permitted procedures for determining the winning team are those approved by the International F.A. Board, namely:

\begin{itemize}
\item away goals rule
\item extra time
\item kicks from the penalty mark
\item \simplify{(new) }extended kicks from the penalty mark
\end{itemize}


{\bfseries Goal-line technology (GLT) (physical competition only) }

\headlinebox

GLT systems may be used for the purpose of verifying whether a goal has been scored to support the referee{\textquoteright}s decision. The use of GLT must be stipulated in the respective competition rules.
