\clearpage
\sffamily
{\bfseries\color[rgb]{0.4,0.4,0.4}{Law 13 -- Free Kicks} }
\phantomsection
\addcontentsline{toc}{subsection}{Law 13 -- Free Kicks}

\bigskip

{\bfseries Types of free kick}

\headlinebox

Free kicks are either direct or indirect.

\bigskip

{\bfseries The direct free kick }

\headlinebox

Ball enters the goal:

\begin{itemize}
\item if a direct free kick is kicked directly into the
opponents{\textquoteright} goal, a goal is awarded
\item if a direct free kick is kicked directly into the team's own goal, a corner kick is awarded to the opposing team
\end{itemize}

\bigskip

{\bfseries The indirect free kick}

\headlinebox

\greyed{(suspended:
Signal

The referee indicates an indirect free kick by raising his arm above his head. He maintains his arm in that position until the kick has been taken and the ball has touched another player or goes out of play.)}


\bigskip

Ball enters the goal

A goal can be scored only if the ball is either kicked and clearly moves as
determined by the referee\footnote{In the virtual competition, a move of at least 5 cm is considered clearly moving.} or has been touched by another player before being
kicked into the goal
\greyed{(replaces: subsequently touches another player before it enters the goal)}:

\begin{itemize}
\item if an indirect free kick is kicked directly into the
opponents' goal, a goal kick is awarded
\item if an indirect free kick is kicked directly into the
team's own goal, a corner kick is awarded to the
opposing team
\end{itemize}

\bigskip

{\bfseries Procedure}

\headlinebox

All free kicks are taken from the place where the offence occurred, except:
\begin{itemize}
\item indirect free kicks to the attacking team for an offence inside the
  opponents' penalty area are taken from the nearest point
  on the penalty area line which runs parallel to the goal
  line.
\item in the physical competition, free kicks to the defending team in their goal area may be taken from anywhere in that area

\end{itemize}

The ball:
\begin{itemize}
\item must be stationary \greyed{(suspended: and the kicker must not touch the ball again until it has touched another player)}
\item is in play when it is kicked and clearly moves as determined by the
      referee\footnote{In the virtual competition, a move of at least 5 cm is considered clearly moving.} except for a free kick to the defending team in their penalty area
      where the ball is in play when it is kicked directly out of the penalty area.
      \simplify{(new): }In both cases the the ball is also considered in play 10 seconds
      after the referee gave the signal.
\end{itemize}

Until the ball is in play all opponents must remain:
\begin{itemize}
\item at least 0.75m for KidSize and 1.5m for AdultSize
      \greyed{(replaces: 9.15 m (10 yds))} from the ball until it is in play,
      unless they are on their own goal line between the goalposts 
\item outside the penalty area for free kicks inside the opponents' penalty area
\end{itemize}

The referee blows the whistle (physical competition only), announces 'Free-Kick' blue or
red and communicates ``Direct / Indirect Free Kick'' Blue/Red to the
  GameController or the assistant referee operating the GameController.
The referee places the ball depending on the call and announces
``Free Kick Ready'' and communicates
``Prepare Direct / Indirect Free Kick'' Blue/Red to the GameController or the assistant referee operating the GameController.
The player taking the free kick has up to 30 seconds to position themselves for
the free kick.
In the virtual competition, any player from the team taking the free kick may announce that the
player is ready to take the free kick at any point by sending a
message to the GameController. In the physical competition, the robot handler of the team taking the free kick can announce to the referee that the player is ready to take the free kick.

Players are guaranteed at least 15 seconds to move away from the ball.
They may take up to 30 seconds if the team taking the free kick has not
announced their robot is ready to take the kick off.
Any opponent robot still illegally positioned is considered as an incapable
player and must be removed from the field for 30 seconds removal penalty.
The referee may decide to execute the free kick before 15 seconds have passed if
the team taking the free kick have announced their robot is ready and if no
opponent is illegally positioned.
Once the free kick can be executed, the referee blows the whistle (physical competition only) and communicates ``Execute Direct / Indirect Free Kick'' Blue/Red to the GameController or the assistant referee operating the GameController.

\bigskip

{\bfseries Infringements and sanctions}

\headlinebox

If, when a free kick is taken, an opponent is closer to the ball than the required distance:

\begin{itemize}
\item the opponent receives a 30 second removal penalty \greyed{(replaces: the kick is retaken)}
\end{itemize}


    In a physical competition, if when a free kick is taken by the defending team from inside its own
    penalty area, the ball is not kicked directly out of the penalty area:

  \begin{itemize}
  \item the kick is retaken \simplify{(new:) }if the goal keeper
      managed to reach the ball within the time frame. Otherwise, the ball is in
      play again.
  \end{itemize}
  \bigskip


\greyed{(suspended:)
\textbf{Free kick taken by a player other than the goalkeeper}

If, after the ball is in play, the kicker touches the ball again (except with his hands) before it has touched another player:

\begin{itemize}
\item an indirect free kick is awarded to the opposing team, to be taken from the place where the infringement occurred (see Law 13 -- Position of free kick)
\end{itemize}

\bigskip

(suspended:) If, after the ball is in play, the kicker deliberately handles the ball before it has touched another player:

\begin{itemize}
\item a direct free kick is awarded to the opposing team, to be taken from the place where the infringement occurred (see Law 13 -- Position of free kick)
\item a penalty kick is awarded if the infringement occurred inside the kicker's penalty area
\end{itemize}

\bigskip

(suspended:) \textbf{Free kick taken by the goalkeeper}

If, after the ball is in play, the goalkeeper touches the ball again (except with his hands), before it has touched another player:

\begin{itemize}
\item an indirect free kick is awarded to the opposing team, to be taken from the place where the infringement occurred (see Law 13 -- Position of free kick)
\end{itemize}

\bigskip

(suspended:) If, after the ball is in play, the goalkeeper deliberately handles the ball before it has touched another player:

\begin{itemize}
\item a direct free kick is awarded to the opposing team if the infringement occurred outside the goalkeeper's penalty area, to be taken from the place where the infringement occurred (see Law 13 -- Position of free kick)
\item an indirect free kick is awarded to the opposing team if the infringement occurred inside the goalkeeper's penalty area, to be taken from the place where the infringement occurred (see Law 13 -- Position of free kick)
\end{itemize}

\bigskip
}

\simplify{(new) }If a free kick was awarded to team A and any player of team A
touches the ball before the referee announced the execution of the free kick:

\begin{itemize}
\item The ball is in play.
\item The player touching the ball received a warning. For the second warning, the player received a yellow card. For the fourth warning, the player receives a second yellow card.
\end{itemize}

\bigskip

\simplify{(new) }If a free kick was awarded to team A and any player of team B
touches the ball before the referee announced the execution of the free kick:


\begin{itemize}
\item The free kick is retaken.
\item The player touching the ball received a warning. For the second warning, the player received a yellow card. For the fourth warning, the player receives a second yellow card.
\end{itemize}
