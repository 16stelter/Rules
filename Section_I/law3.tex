\clearpage
\sffamily

{\bfseries\color[rgb]{0.4,0.4,0.4}
Law 3 -- The Players}
\phantomsection
\addcontentsline{toc}{subsection}{Law 3 -- The Players}


\bigskip

{\bfseries Number of Players}

\headlinebox

A match is played by two teams, each consisting of not more than four players in KidSize and not more than two players in AdultSize,
one of whom must be designated as goalkeeper.
A match may not start if either team consists of less than one player.
In a virtual competition it is considered a forfeit if a team does not provide any software to play with for an upcoming game.
If a team has not at least one player (who may be incapable to play) at the side of the field,
it is considered a forfeit in a physical competition.

\greyed{
(replaces: A match is played by two teams, each consisting of not more than eleven players, one of whom is the goalkeeper. A match may not start if either team consists of fewer than seven players.)}

\bigskip

{\bfseries Number of substitutions (physical competition only)}

\headlinebox
 
{\bfseries Official competitions }

Up to a maximum of two \greyed{(replaces: three)} substitutes may be used in any match played in an official competition organised under the auspices of FIFA, the confederations or the member associations.

The rules of the competition must state how many substitutes may be nominated, from two \greyed{(replaces: three)} up to a maximum of twelve.

\bigskip

\greyed{
(suspended:
\simplify{
{\bfseries Other matches }

In national ``A'' team matches, up to a maximum of six substitutes may be used.

\bigskip

In all other matches, a greater number of substitutes may be used provided that:

\begin{itemize}
\item the teams concerned reach agreement on a maximum number 
\item the referee is informed before the match
\end{itemize}

If the referee is not informed, or if no agreement is reached before the match, no more than six substitutes are allowed.}

\bigskip
}

{\bfseries Substitution procedure (physical competition only)}

\headlinebox

In all matches, the names of the substitutes must be given to the referee prior to the start of the match. Any substitute whose name is not given to the referee at this time may not take part in the match.

\bigskip

To replace a player with a substitute, the following conditions must be observed:

\begin{itemize}
\item the referee must be informed before any proposed substitution is made
\item the substitute only enters the field of play after the player being replaced has left and after receiving a signal from the referee
\item the substitute only enters the field of play at the penalty mark of the player's own half \greyed{
(replaces: the halfway line)} and during a stoppage in the match
\item the substitution is completed when a substitute enters the field of play
\item from that moment, the substitute becomes a player and the player he has replaced becomes a substituted player
\item \greyed{(suspended: the substituted player takes no further part in the match)}
\item all substitutes are subject to the authority and jurisdiction of the referee, whether called upon to play or not
\end{itemize}

{\bfseries Changing the goalkeeper}

\headlinebox

Any of the other players may change places with the goalkeeper, provided that:

\begin{itemize}
      \item the GameController (virtual competition) or the referee (physical competition) is informed before the change is made
      \item the change is requested during a stoppage in the match
\end{itemize}

{\bfseries Infringements and sanctions (physical competition only)}

\headlinebox

If a substitute or substituted player or a team official enters the field of play
without the referee's permission:

\begin{itemize}
\item the referee stops play (although not immediately if the substitute or substituted player does not interfere with play)
\item the referee cautions him for unsporting behaviour and orders him to leave the field of play
\item if the referee has stopped play, it is restarted with an direct free kick
      for the opposing team from the position of the ball at the time of the
      stoppage (see Law 13 -- Position of free kick)
\end{itemize}

\bigskip

If a named substitute enters the field of play instead of a named player at the start of the match and the referee is not informed of this change:

\begin{itemize}
\item the referee allows the named substitute to continue the match
\item no disciplinary sanction is taken against the named substitute
\item the number of substitutions allowed by the offending team is not reduced
\item the referee reports the incident to the appropriate authorities
\end{itemize}

\bigskip

If a player changes places with the goalkeeper without the
referee's permission before the change is made:

\begin{itemize}
\item the referee allows play to continue
\item the referee cautions the players concerned when the ball is next out of play
\end{itemize}

\bigskip

In the event of any other infringements of this Law:

\begin{itemize}
\item the players concerned are cautioned
\item the match is restarted with an indirect free kick, to be taken by a player of the opposing team from the position of the ball at the time of the stoppage (see Law 13 -- Position of free kick)
\end{itemize}

\bigskip

{\bfseries Players and substitutes sent off (physical competition only)}

\headlinebox

A player who has been sent off before the kick-off may be replaced only by one of the named substitutes.

\bigskip

A named substitute who has been sent off, either before the kick-off or after play has started, may not be replaced.
