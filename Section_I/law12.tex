\clearpage
\sffamily
{\bfseries\color[rgb]{0.4,0.4,0.4}{Law 12 -- Fouls and Misconduct} }
\phantomsection
\addcontentsline{toc}{section}{Law 12 -- Fouls and Misconduct}

\bigskip

Fouls and misconduct are penalised as follows:

\bigskip

{\bfseries Direct free kick }

\headlinebox

A direct free kick is awarded to the opposing team if a player commits any of the following offences to a player of the opposing team:
\begin{itemize}
\item uses forceful contact that significantly destabilizes a player, such that walking and/or kicking is impeded. Examples for forceful contacts include falling into another player or walking carelessly into another player at significant speed.
\item walks into another player for 4 to 5 seconds (even a fallen or getting up player), even if the 'force to push' is minimal.
\end{itemize}

A free kick is not awarded if one of the following exceptions occurs: 
\begin{itemize}
\item The player committing the offence is stationary, including a player that is kicking, provided that the ball was close enough where a kick could have succeeded at the start of the kick motion.
\item The player committing the offence is currently getting up.
\item The player committing the offence is the current goal keeper and is currently looking at or chasing the ball in it's own penalty area.
\item Front to front contact between players with the ball between them does not lead to a free kick, unless one player walks at a significantly higher speed or with significantly more force that is impossible to stand for the other player.
\item Any player proceeding to the ball whose side (i. e. arm, shoulder etc.)  makes contact with another player is not committing an offence, even if the second player is not proceeding to the ball.
\item A player that had an offence committed against himself can not simultaneously be called for a free kick offence himself.
\end{itemize}


\greyed{
(replaces: A direct free kick is awarded to the opposing team if a player commits any of the following seven offences in a manner considered by the referee to be careless, reckless or using excessive force:

\begin{itemize}
\item kicks or attempts to kick an opponent
\item trips or attempts to trip an opponent
\item jumps at an opponent 
\item charges an opponent 
\item strikes or attempts to strike an opponent
\item pushes an opponent 
\item tackles an opponent) 
\end{itemize}}

\bigskip


A direct free kick is also awarded to the opposing team if a player commits any of the following three offences: 

\begin{itemize}
\item holds an opponent
\item spits at an opponent
\item handles the ball deliberately (except for the goalkeeper within his own penalty area)
\item (new:) holds the ball for more than 1 second in a way that the ball cannot be removed from the player (a goal keeper may hold the ball up to 6 seconds on the ground or 10 seconds lifted up with one or both hands). More than half of the ball's volume must be outside the convex hull of the player, projected to the ground, for the ball to be considered removable. If the ball enters the convex hull repeatedly, it must be removable in between for the majority of the time. If more than one player of a team is in the vicinity of the ball, the convex hull is taken around all the player of a team, which prevent removal of the ball.
\end{itemize}

\bigskip

(new:) If an offense did not happen within a radius of approx. 1 m around the current ball position, the direct free kick is replaced by a removal penalty.

\bigskip


A direct free kick is taken from the place where the offence occurred (see Law 13 -- Position of free kick). (new:) If moving the ball to the place where the offence occurred would be to the disadvantage of the team to which the free kick is awarded, the referee allows play to continue.



\bigskip

{\bfseries Penalty kick}

\headlinebox

A penalty kick (new) as defined by Law 14 is awarded if any of the above ten offences is committed by a player inside his own penalty area, irrespective of the position of the ball, provided it is in play.


\bigskip

{\bfseries Indirect free kick}

\headlinebox

An indirect free kick is awarded to the opposing team if a goalkeeper, inside his own penalty area, commits any of the following four offences: 

\begin{itemize}
\item controls the ball with his hands for more than ten seconds before releasing it from his possession
\item touches the ball again with his hands after he has released it from his possession and before it has touched another player
\item touches the ball with his hands after it has been deliberately kicked to him by a team-mate
\item touches the ball with his hands after he has received it directly from a throw-in taken by a team-mate
\end{itemize}

\bigskip

An indirect free kick is also awarded to the opposing team if, in the opinion of the referee, a player:

\begin{itemize}
\item plays in a dangerous manner
\item impedes the progress of an opponent
\item prevents the goalkeeper from releasing the ball from his hands
\item commits any other offence, not previously mentioned in Law 12, for which play is stopped to caution or send off a player
\end{itemize}

\bigskip

(new:) If an offense did not happen within a radius of approx. 1 m around the current ball position, the indirect free kick is replaced by a removal penalty.

\bigskip

The indirect free kick is taken from the place where the offence occurred (see Law 13 -- Position of free kick). (new:) If moving the ball to the place where the offence occurred would be to the disadvantage of the team to which the free kick is awarded, the referee allows play to continue.

\bigskip

{\bfseries Disciplinary sanctions}

\headlinebox

The yellow card is used to communicate that a player, substitute or substituted player has been cautioned.

\bigskip

The red card is used to communicate that a player, substitute or substituted player has been sent off.

\bigskip

Only a player, substitute or substituted player may be shown the red or yellow card.

\bigskip

The referee has the authority to take disciplinary sanctions from the moment he enters the field of play until he leaves the field of play after the final whistle. 

\bigskip

A player who commits a cautionable or sending-off offence, either on or off the field of play, whether directed towards an opponent, a team-mate, the referee, an assistant referee or any other person, is disciplined according to the nature of the offence committed.

\bigskip

{\bfseries Cautionable offences }

\headlinebox

A player is cautioned and shown the yellow card if he commits any of the following seven offences: 

\begin{itemize}
\item unsporting behaviour
\item dissent by word or action
\item persistent infringement of the Laws of the Game
\item delaying the restart of play
\item \greyed{(suspended: failure to respect the required distance when play is restarted with a corner kick, free kick or throw-in)}
\item entering or re-entering the field of play without the referee's permission
\item \greyed{(suspended: deliberately leaving the field of play without the referee's permission)}
\end{itemize}

\bigskip

A substitute or substituted player is cautioned if he commits any of the following three offences: 

\begin{itemize}
\item unsporting behaviour
\item dissent by word or action 
\item delaying the restart of play
\end{itemize}

{\bfseries Sending-off offences}

\headlinebox

A player, substitute or substituted player is sent off if he commits any of the following seven offences:

\begin{itemize}
\item serious foul play
\item violent conduct
\item spitting at an opponent or any other person
\item denying the opposing team a goal or an obvious goalscoring opportunity by deliberately handling the ball (this does not apply to a goalkeeper within his own penalty area)
\item \greyed{(suspended: denying an obvious goalscoring opportunity to an opponent moving towards the player's goal by an offence punishable by a free kick or a penalty kick)}
\item using offensive, insulting or abusive language and/or gestures
\item receiving a second caution in the same match
\end{itemize}

\bigskip

A player, substitute or substituted player who has been sent off must leave the vicinity of the field of play and the technical area.