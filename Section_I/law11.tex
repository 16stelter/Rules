\clearpage
\sffamily
{\color[rgb]{0.4,0.4,0.4}
\textbf{(suspended: LAW 11 -- Offside} }
\addcontentsline{toc}{section}{Law 11 -- Offside (suspended)}

\bigskip

{\color[rgb]{0.4,0.4,0.4}\textbf{Offside position}

\headlinebox

It is not an offence in itself to be in an offside position. A player is in an offside position if:

\begin{itemize}
\item he is nearer to his opponents{\textquoteright} goal line than both the ball and the second-last opponent
\end{itemize}

\bigskip

A player is not in an offside position if:

\begin{itemize}
\item he is in his own half of the field of play or
\item he is level with the second-last opponent or
\item he is level with the last two opponents
\end{itemize}
}

{\color[rgb]{0.4,0.4,0.4}\textbf{Offence}

\headlinebox

A player in an offside position is only penalised if, at the moment the ball touches or is played by one of his team, he is, in the opinion of the referee, involved in active play by:

\begin{itemize}
\item interfering with play or 
\item interfering with an opponent or 
\item gaining an advantage by being in that position
\end{itemize}
}

{\color[rgb]{0.4,0.4,0.4} \textbf{No offence}

\headlinebox

There is no offside offence if a player receives the ball directly from:

\begin{itemize}
\item a goal kick 
\item a throw-in 
\item a corner kick 
\end{itemize}
Infringements and sanctions \\
In the event of an offside offence, the referee awards an indirect free kick to the opposing team to be taken from the place where the infringement occurred (see Law 13 -- Position of free kick).) 

}