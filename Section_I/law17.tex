\clearpage
\sffamily
{\bfseries\color[rgb]{0.4,0.4,0.4}
LAW 17 -- THE CORNER KICK}
\addcontentsline{toc}{section}{LAW 17 -- THE CORNER KICK}


\bigskip

A corner kick is a method of restarting play.

\bigskip

A corner kick is awarded when the whole of the ball passes over the goal
line, either on the ground or in the air, having last touched a player
of the defending team, and a goal is not scored in accordance with Law
10.

\bigskip

{\color[rgb]{0.4,0.4,0.4}(suspended: A goal may be scored directly
from a corner kick, but only against the opposing team.)}

\bigskip

{\bfseries Procedure}

\headlinebox

If the ball leaves the field it will be replaced on the field by the
referee or an assistant referee. There is no stoppage in play. 

If the whole of the ball passes over a goal line then the ball will be
replaced back on the field according to the following rules: 

\begin{itemize}
\item If the referee cannot determine which robot was the last to touch
the ball before it left the field, then the ball is replaced in about 1
meter distance from the corner of the field
\item If the ball was last touched by the defensive team then the ball
is replaced in a distance of a about the goal area length from the
closest corner of the field
\end{itemize}

Balls are deemed to be out based on the team that last touched the ball,
irrespective of who actually kicked the ball. 

\bigskip

{\color[rgb]{0.4,0.4,0.4}
(replaces: 

\begin{itemize}
\item The ball must be placed inside the corner arc nearest to the point where
the ball crossed the goal line 
\item The corner flagpost must not be moved
\item Opponents must remain at least 1 m from the corner arc until the ball is
in play (replaces: Opponents must remain at least 9.15 m (10 yds) from
the corner arc until the ball is in play )
\item The ball must be kicked by a player of the attacking team
\item The ball is in play when it is kicked and moves
\item The kicker must not play the ball again until it has touched another
player)
\end{itemize}
}

\bigskip

{\bfseries Infringements and sanctions}

\headlinebox

{\color[rgb]{0.4,0.4,0.4}
(suspended: Corner kick taken by a player other than the goalkeeper

If, after the ball is in play, the kicker touches the ball again (except
with his hands) before it has touched another player:

\begin{itemize}
\item an indirect free kick is awarded to the opposing team, to be taken from
the place where the infringement occurred (see Law 13 -- Position of
free kick)
\end{itemize}

\bigskip

If, after the ball is in play, the kicker deliberately handles the ball
before it has touched another player:

\begin{itemize}
\item a direct free kick is awarded to the opposing team, to be taken from the
place where the infringement occurred (see Law 13 -- Position of free
kick) 
\item a penalty kick is awarded if the infringement occurred inside the
kicker{\textquoteright}s penalty area 
\end{itemize}

\bigskip

Corner kick taken by the goalkeeper

If, after the ball is in play, the goalkeeper touches the ball again
(except with his hands) before it has touched another player:

\begin{itemize}
\item an indirect free kick is awarded to the opposing team, to be taken from
the place where the infringement occurred (see Law 13 -- Position of
free kick)
\end{itemize}

\bigskip

If, after the ball is in play, the goalkeeper deliberately handles the
ball before it has touched another player:

\begin{itemize}
\item a direct free kick is awarded to the opposing team if the infringement
occurred outside the goalkeeper{\textquoteright}s penalty area, to be
taken from the place where the infringement occurred (see Law 13 --
Position of free kick)
\item an indirect free kick is awarded to the opposing team if the
infringement occurred inside the goalkeeper{\textquoteright}s penalty
area, to be taken from the place where the infringement occurred (see
Law 13 -- Position of free kick)
\end{itemize}

\bigskip

In the event of any other infringement:

\begin{itemize}
\item the kick is retaken)
\end{itemize}
}